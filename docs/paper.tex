

\documentclass[10pt,twocolumn]{article}
\usepackage{usenix-2e}
\usepackage[compact,small]{titlesec} % ,small
%\titlespacing{\section}{0pt}{2ex}{.5ex}
%\titlespacing{\subsection}{0pt}{1.5ex}{.5ex}

%\usepackage{paralist}
%\setlength{\plitemsep}{.75ex}
%\setlength{\pltopsep}{.75ex}

\setlength{\belowcaptionskip}{.5ex}
\setlength{\textfloatsep}{3ex}

\usepackage[english]{babel}
\usepackage{url}
\usepackage{listings}
\usepackage{graphicx}

\usepackage[square,numbers]{natbib}
\setlength{\bibsep}{.5ex}

\lstset{language=C,linewidth=.89\textwidth,print=true,captionpos=b,basicstyle=\ttfamily,breaklines,extendedchars=true,commentstyle=\footnotesize,fontadjust=true}

\begin{document}
\title{\Large \bf The EDelta Algorithm for Linear time, Constant space
Executable Differencing}

\author{
{\rm Jacob Gorm Hansen}
\\Department of Computer Science, University of Copenhagen, Denmark
}

\maketitle

\begin{abstract}
{\small\it Abstract here}
\end{abstract}

\section{Introduction}

Various algorithms for extracting differences between two related binary files
have been described, most prominently zdelta, xdelta, and rsync. These
algorithms achieve good results when differences can be described as insertions
and deletions of substrings, but do no take into account other forms of
systematic changes, such as those resulting from the recompilation of a program
from slightly changed sources. Since the need for propagating software updates,
for instance security patches, to clients with limited network connections, such
as cell phones or modem-connected home computers, is increasing, a specialised
solution to this problem may be beneficial. In addition, Software Configuration
Management systems, such as Vesta~\cite{vesta} may wish to include versioning
build systems which stores all binaries for future reference, and developers
working across wide-area links with limited bandwidth can increase productivity
if they are able to quickly deploy large binaries to machines in remote data
centres.

Different rules may apply when updating software than when updating other types
of files, and various tradeoffs can be made. For instance, if it is known that
only a certain procedure needs to be replaced, theoretically only the compiled
code for this procedure needs to be transmitted, and the original version
replated with this one, or alternatively with a jump instruction to the location
of the new one, in the likely case that its definition has grown in size. However,
patching procedure definitions like this will result in the target file not
matching its server side counterpart byte for byte, and may adversely affect the
performance of the code, due to changes in cache locality.  And while relaxing
the constraint that the resulting file on the target should match the result of
compilation exactly may result in a more efficient delta transfer, as the amount
of changes communicated will be proportional in the amount of changes at the
source code level, subsequent updates, as well as security audits, will become
harder.

Another option, as is common with the BSD UNIX variants, is to simply distribute
changes in source code form, requiring that the source of the previous version
and a suitable suite of compilers be present at the target machine.  However,
this is rarely an option for embedded and small scale desktop systems, with no
capacity to host a suite of build tools, and without operators skilled in
software maintenance at the the source level.

In this paper, we describe the EDelta delta-compression algorithm, a variant of
the linear time, constant space differencing algorithm described by Burns and
Long~\cite{lineartime}. Similar to the Burns and Long algorithm, EDelta trades
some optimality for platform and instruction-set neutrality, run-time
performance and memory efficiency. Competing algorithms are either very
expensive in terms of memory and CPU complexity, as is the case for
BSDiff~\cite{bsdiff}, or specific to a single instruction set family, in the
case of ExeDiff~\cite{exediff} or Microsoft's VPEDiff which is based on
Vulcan~\cite{vulcan}. EDelta also sees increasing competition from non-dedicated
delta compression algorithms, mainly ZDelta, a variation of LZ77. Compared to
the latter, EDelta's advantages are mainly speed, compression ratio for smaller
changes, and the ability to apply changes in-place.

The rest of the paper is laid out as follows: first we examine the ways a binary
file changes when changes are made to the source from which it is compiled. We
then briefly describe the linear algorithm on which EDelta is based, and the
modifications we introduce to make it perform better for executable code. We
evaulate EDelta in comparison to other algorithms for a set of different inputs,
measuring performance both in terms of delta file sizes and execution time for
delta extraction.

\section{Examining the structure of changes}

The original motivation for the work that lead to the implementation of EDelta,
was that existing longest-common-substring delta algorithms such as RSync and
Xdelta often performed poorly when used for updating executables. The author was
doing development work where a fast machine used for compilation of Linux
kernelsn was placed remotely, whereas the debug target machine was present
locally, and a slow ADSL link had to be traversed for each compile-debug cycle.
RSync was used for transmitting updates from the compilation machine to the
local debug target, but any non-trivial change at the source level would result
in very poor compression, because of cascading effects caused by relocations of
code in the executable Linux kernel image. These relocations would confuse the
longest-common-substring (LCS) matcher at the heart of RSync to believe that
large sections of the file had changed, degrading the compression ratio.

With this in mind, we decided to look for a compact way of encoding the
structure of changes to the binary executable, even in the face of relocations.
First we performed a simple experiments using existing textual delta tools to
visualize the structure of changes. We dis-assembled two binary version of the
same program, filtered away any irrelevant information such as line numbers and
identical instruction sequences, and ran the textural ``diff'' command on the
remaining text. The result provided the inspiration for the further development
of EDelta.  Table~\ref{tab:diff} shows the textual differences between the
assembly outputs of two versions of a simple ``hello world'' C program compiled
for PowerPC, with code added between versions. The added code, an added local
variable and a simple conditional, results in a few new instructions, but as the
code has moved in memory, all references have been offset by $\pm 20$ bytes. This
observation motivates us to attempt to code changes not only as byte-range
copies and additions, but also as a range of offsets within a file at which a
numeric value should be added. To an LCS-algorithm, the instruction sequences
will appear different, however it should be clear to the reader that if the
LCS-matcher ignored the non-matching bytes (bold-faced in the table), more
and longer matches would be possible, and remaining differences made up for by
patching changed offsets in a separate phase.


\begin{table*}
\begin{center}

\begin{tabular}{|r|l||r|l|r|}
\hline
Hexadecimal (a) & Mnemonic (a) & Hexadecimal (b) & Mnemonic (b)  & Offset(decimal)\\ 
\hline
\hline
 48 00 04 \bf b1    &   bl 0x1000071c            &          48 00 04 \bf c5  & b 0x10000730            & 20 \\
\hline
 85 a8 07 \bf a8    &   lwzu r13,1960(r8)        &          85 a8 07 \bf bc  & lwzu r13,1980(r8)       & 20 \\
 48 01 06 \bf 78    &   b 0x10010918             &          48 01 06 \bf 8c  & b 0x1001092c            & 20 \\
\hline
 48 01 06 \bf 09    &   bl 0x100108bc            &          48 01 06 \bf 1d  & bl 0x100108d0           & 20 \\
\hline
 88 1e 09 \bf 30    &   lbz r0,2352(r30)         &          88 1e 09 \bf 44  & lbz r0,2372(r30)        & 20 \\
\hline
 81 7f 07 \bf c8    &   lwz r11,1992(r31)        &          81 7f 07 \bf dc  & lwz r11,2012(r31)       & 20 \\
\hline
 90 1f 07 \bf c8    &   stw r0,1992(r31)         &          90 1f 07 \bf dc  & stw r0,2012(r31)        & 20 \\
\hline
 81 7f 07 \bf c8    &   lwz r11,1992(r31)        &          81 7f 07 \bf dc  & lwz r11,2012(r31)       & 20 \\
\hline
 98 1e 09 \bf 30    &   stb r0,2352(r30)         &          98 1e 09 \bf 44  & stb r0,2372(r30)        & 20 \\
\hline
 80 0b 08 \bf b8    &   lwz r0,2232(r11)         &          80 0b 08 \bf cc  & lwz r0,2252(r11)        & 20 \\
\hline
 38 6b 08 \bf b8    &   addi r3,r11,2232         &          38 6b 08 \bf cc  & addi r3,r11,2252        & 20 \\
\hline
                    &                            &          38 00 00 02  & li r0,2                     & \\
                    &                            &          90 1f 00 08  & stw r0,8(r31)               & \\
                    &                            &          80 1f 00 08  & lwz r0,8(r31)               & \\
                    &                            &          2c 00 00 01  & cmpwi r0,1                  & \\
                    &                            &          40 81 00 14  & ble 0x10000410                     & \\
\hline
 38 69 07 \bf bc    &   addi r3,r9,1980          &          38 69 07 \bf d0  & addi r3,r9,2000         & 20 \\
\hline
 4b ff \bf fe \bf 11    &   bl 0x10000258        &          4b ff \bf fd \bf fd  & bl 0x10000258               & -20 \\
\hline
 39 29 08 \bf ac    &   addi r9,r9,2220          &          39 29 08 \bf c0  & addi r9,r9,2240         & 20 \\
\hline
 4b ff fb \bf 51    &   bl 0x100002e4            &          4b ff fb \bf 3d  & bl 0x100002e4           & -20 \\
\hline
\end{tabular}

\end{center}
\caption{ \label{tab:diff} Diff}
\end{table*}

\section{Proposed algorithm}

Traditional Longest-Common-Substring (LCS) delta algorithms encode the new
version as a number of strings to be copied from the original file, as well as a
number of new strings to be added. Since copy operations can be represented as
(source, destination, length) tuples of constant size, the size of the delta
file will usually be dominated by the length of the added strings. If we can
expand the lengths of the copied strings and reduce the lengths of the
additions, the delta file size will be reduced as a result. Based on the
observersations made above, we hypothesise that if we expand the length of
copied strings by allowing for constant-sized holes to occur when matching
ranges, and later patching these in a separate pass, we may be able to obtain a
smaller overall delta file size, as the holes in matching ranges will represent
an often occuring offset change, and thus may be effeciently represented.

Our algorithm is a straight-forward specialisation of the algorithm proposed in
\cite{alinear}. While not always finding the optimal delta, this algorithm has
the desirable property that its complexity is linear ($O(n)$) in the size of the
larger of its inputs. It works by computing a rolling hash for each offset
in base and version files, looking these up in a hash table, and comparing
ranges when matching hashes are found. When matching ranges are encountered,
these are encoded in the delta file as an addition of the data leading up to the
match, as well as a copy of the extent of the match in the base file, to its new
location in the version file. 

Given the characteristics noted from our examination of the hello world example
above, we attempt to improve the amount of copy operations, and thus reuse
between versions, by allowing holes up to the size of an integer (currently 32
bits) in our matching ranges, deferring the repair of these to a later stage.
Our assumption is that holes appearing within matching ranges will frequently be
due to instructions referencing relocated symbols, and that the result of
subtracting the new and old values of the mismatching integers will usually
represent a globally changed offset, and thus be occur with high frequency.

Table \ref{table:hello-sizes} shows the size of the two versions of the hello
world executable, compiled for the PowerPC architecture and stripped of
debugging symbols. 

\section{Related Work}

Due to the increasing need for timely dispersal of updates for operating systems
and applications exposed to the perils of the Internet, there has been a growing
interest in delta compression algorithms for executables. All algorithms do
essentially the same thing--convert two executables $b$ and $v$ into a patch $p$
such that $v = b+p$--but the design space contains various tradeoffs, with
regards to types of input accepted and running time and space demands for delta
extraction and patching.

\begin{table*}
\begin{center}
\begin{tabular}{|r|c|c|c|}
\hline

                        & ExeDiff        &      BSDiff    & EDelta     \\

\hline
Run-time complexity     & O(n)           &   O((n+m) log n) & O(min(n,m))   \\  
Space complexity        &                & max(17*n,9*n+m)+O(1) &            \\
Instruction-set neutral & No             &  Yes           & Yes        \\
\hline

\end{tabular}
\caption{ \label{tab:comparisons} Comparisons}
\end{center}
\end{table*}



\section{Acknowledgments}

{\footnotesize
\citeindexfalse
\bibliographystyle{unsrt}
\bibliography{biblio}
\citeindextrue
}
\end{document}
